\documentclass[oribibl]{llncs}

\usepackage[T1]{fontenc} \usepackage[utf8]{inputenc} \usepackage[english]{babel}

\usepackage{cite} %Vereinfachung der Referenzen: [1], [2], [3] => [1-3]
\usepackage{makeidx}         % allows index generation

\usepackage{graphicx} \usepackage[cmex10]{amsmath} %mathamatische Formeln 
\usepackage{wrapfig}
\usepackage{placeins}
\usepackage{mathtools}
\usepackage{listings}
\lstset{basicstyle=\small}

\begin{document} \title{Tutorial for usage of the RCLL with the Fawkes Robotics Framework in the Gazebo simulation and YAGI}

\author{Nicolas Limpert} \institute{Fachhochschule Aachen - University of Applied Sciences \and MASCOR Institute}

\maketitle

\begin{abstract}
        This document is supposed to serve as a tutorial for getting started with the software stack provided by the Team Carologistics\footnote{http://www.carologistics.org} in 2015.\\
        Fawkes\footnote{http://www.fawkesrobotics.org} is a robotics framework and fully capable of handling the RoboCup Logistics League (RCLL) \footnote{http://www.robocup-logistics.org}.\\
        The Fawkes implementation meant for the Festo Robotino robots additionally make use of an agent written in the C-Language Integrated Production System (CLIPS).\\
        In this tutorial we instead want to make use of YAGI (Yet Another Golog Interpreter)\footnote{http://yagi.ist.tugraz.at/}.
\end{abstract}

\end{document}

